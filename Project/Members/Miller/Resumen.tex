\documentclass[a4 paper,12pt]{article}
\usepackage[utf8]{inputenc}
%\usepackage{url}
\usepackage{hyperref}
\usepackage{color}
\begin{document}
	\section*{RESUMEN}
	\paragraph{Motivación/Objetivo:}
	En el ámbito de la estadística y probabilidad existe controversia sobre cuándo utilizar la mediana como medida de tendencia central. El objetivo de este artículo es valorar mediante un programa en R su utilidad.
	\paragraph{Metodología:}
	Para poder llevar a cabo esta investigación, se han revisado artículos científicos similares y consultado la base de datos de “{\color{blue}\url{data.worldbank.org}}” World Bank Open Data, ScienceDirect y otros de donde extrajimos la muestra para la experimentación.
	\paragraph{Experimentación/Conclusión:}
	Después de haber probado diferentes métodos para el cálculo aleatorizado de la mediana, llamamos “randomquicksort” a nuestro programa más eficiente.
	\paragraph{Palabras Claves:}
	Tendencia central, mediana, algoritmo aleatorizado, randomquicksort.
\end{document}